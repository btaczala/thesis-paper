% Wprowadzenie, rozdzia� 1 
\chapter{Wprowadzenie} 
\label{Rozdzia� 1 }
\lhead{Rozdzia� 1. \emph{Tematyka pracy }} 


\section{Tematyka pracy}

Welcome to this ``\LaTeX{} Thesis Template'', a beautiful and easy to use template for writing a thesis using the \LaTeX{} typesetting system.

If you are writing a thesis (or will be in the future) and its subject is technical or mathematical (though it doesn't have to be), then creating it in \LaTeX{} is highly recommended as a way to make sure you can just get down to the essential writing without having to worry over formatting or wasting time arguing with your word processor.

\LaTeX{} is easily able to professionally typeset documents that run to hundreds or thousands of pages long. With simple mark-up commands, it automatically sets out the table of contents, margins, page headers and footers and keeps the formatting consistent and beautiful. One of its main strengths is the way it can easily typeset mathematics, even \emph{heavy} mathematics. Even if those equations are the most horribly twisted and most difficult mathematical problems that can only be solved on a super-computer, you can at least count on \LaTeX{} to make them look stunning.


\section{Motywacja}

\LaTeX{} is not a WYSIWYG (What You See is What You Get) program, unlike word processors such as Microsoft Word or Corel WordPerfect. Instead, a document written for \LaTeX{} is actually a simple, plain text file that contains \emph{no formatting}. You tell \LaTeX{} how you want the formatting in the finished document by writing in simple commands amongst the text, for example, if I want to use \emph{italic text for emphasis}, I write the `$\backslash$\texttt{emph}\{\}' command and put the text I want in italics in between the curly braces. This means that \LaTeX{} is a ``mark-up'' language, very much like HTML.



\section{Cel pracy}

If you are familiar with \LaTeX{}, then you can familiarise yourself with the contents of the \href{http://www.sunilpatel.co.uk/files/Thesis\%20Template.zip}{Zip file} and the directory structure and then place your own information into the `\texttt{Thesis.cls}' file. Section \ref{FillingFile} on page \pageref{FillingFile} tells you how to do this. Make sure you read section \ref{ThesisConventions} about thesis conventions to get the most out of this template and then get started with the `\texttt{Thesis.tex}' file straightaway.

If you are new to \LaTeX{} it is recommended that you carry on reading through the rest of the information in this document.

\section{Realizacja}

If you are familiar with \LaTeX{}, then you can familiarise yourself with the contents of the \href{http://www.sunilpatel.co.uk/files/Thesis\%20Template.zip}{Zip file} and the directory structure and then place your own information into the `\texttt{Thesis.cls}' file. Section \ref{FillingFile} on page \pageref{FillingFile} tells you how to do this. Make sure you read section \ref{ThesisConventions} about thesis conventions to get the most out of this template and then get started with the `\texttt{Thesis.tex}' file straightaway.

If you are new to \LaTeX{} it is recommended that you carry on reading through the rest of the information in this document.

\section{Implementacja}

If you are familiar with \LaTeX{}, then you can familiarise yourself with the contents of the \href{http://www.sunilpatel.co.uk/files/Thesis\%20Template.zip}{Zip file} and the directory structure and then place your own information into the `\texttt{Thesis.cls}' file. Section \ref{FillingFile} on page \pageref{FillingFile} tells you how to do this. Make sure you read section \ref{ThesisConventions} about thesis conventions to get the most out of this template and then get started with the `\texttt{Thesis.tex}' file straightaway.

If you are new to \LaTeX{} it is recommended that you carry on reading through the rest of the information in this document.


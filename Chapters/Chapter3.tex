% Wprowadzenie, rozdzia� 1 
\chapter{Cz�� praktyczna} 
\label{Rozdzia� 3 }
\lhead{Rozdzia� 3. \emph{Cz�� praktyczna }} 


\section{Opis za�o�e� programu}

\subsection{Co ma program robi�, w jakich �rodowiska dzia�a�. }

\section{Opis technologi u�ytych przy tworzeniu oprogramowania}
\subsection{J�zyk, �rodowisko}
\subsection{Opis bibliotek zewn�trznych}

\section{Struktura programu}
\subsection{Opis modu��w u�ytych w programie}
\subsubsection{Szczeg�owy opis biblioteki fl ( jakie funkcje itd ) }
\subsubsection{Opis parsera matematycznego}
\subsubsection{Opis biblioteki do rysowania funkcji}
\subsection{Jak to si� wszystko zaz�bia ( rysunki pewnie ) }


\section{Opis mo�liwo�ci programu }

\section{Przyk�adowe u�ycie}


% Wprowadzenie, rozdzia� 1 
\chapter{Rozwi�zanie przyk�adowego zadania} 
\label{Rozdzia� 4 }
\lhead{Rozdzia� 4. \emph{Rozwi�zanie przyk�adowego zadania }} 


\section{Postawienie zadania}

Zadaniem kt�ra zostanie rozwi�zane za pomoc� programu b�d�cego podmiotem pracy
magisterskiej, jest problem "dw�ch biznes�w". Jest to problem decyzyjny, w
kt�rym decydent podejmuje jedno z dw�ch informacji przy braku informacji
zupe�nej. 
Problem dw�ch biznes�w brzmi nast�puj�co:
\newline

\begin{quote}
W biznesie B1 mo�na zarobi� od 0 do 10 milion�w z�otych. Prawdopodobie�stwo
dochodu bliskiego 10 milion�w z�otych jest du�e. 
W biznesie B2 mo�na zarobi� od 0 do 20 milion�w z�otych. Prawdopodobie�stwo
dochodu bliskiego 20 milion�w z�otych jest jednak ma�e. \newline Decyzja: W
kt�ry z biznes�w zainwestowa�?
\end{quote}
    


\section{Identyfikacja funkcji g�sto�ci }
\subsection{Definicja splotu funkcji}
\subsubsection[]{Kwantyfikator "du�e"}
\subsubsection[]{Kwantyfikator "ma�e"}
\subsubsection[]{Kwantyfikator "bliskie 10 mln"}
\subsubsection[]{Kwantyfikator "nie bliskie 10 mln"}
\subsubsection[]{Kwantyfikator "bliskie 20 mln"}
\subsubsection[]{Kwantyfikator "nie bliskie 20 mln"}

\section{Rozwi�zanie problemu}

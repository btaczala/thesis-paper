% Wprowadzenie, rozdzia� 1 
\chapter{Cz�� teoretyczna} 
\label{Rozdzia� 2 }
\lhead{Rozdzia� 2. \emph{Cz�� teoretyczna }} 


\section{Wyja�nienie podstawowych zagadnie� z zakresu statystyki matematycznej}

\section{Funkcja g�sto�ci oraz jej w�a�ciwo�ci}

\section{Czym jest splot funkcji oraz jak� rol� pe�ni w statystyce matematycznej}
\subsection{Definicja splotu funkcji}
\subsection{W�a�no�ci splotu funkcji}
\subsection{R�ne sploty funkcji, oraz ich wykresy}

\section{Wyja�nienie podstawowych zagadnie� z zakresu matematyki rozmytej}
\subsection{nCzy s� kwantyfikatory lingwistyczne oraz na czym polega rozmywanie wiedzy}
\subsection{Dlaczego kwantyfikator lingwistyczny mo�na przedstawi� jako funkcj� rozk�adu g�sto�ci prawdopodobie�stwa?}

\section{Centralne twierdzenie graniczne}

